\documentclass[article,nojss]{jss}
\usepackage{amsmath,amsfonts,amssymb}
\usepackage{bm} % Bold Math package
\usepackage{url}
\def\command#1{{\it #1}}
\newfont{\tit}{cmss10 scaled\magstep1}
\def\ti#1{{\tit #1}}

\author{Esam Mahdi\\ University of
               Western Ontario \And
         A. Ian McLeod\\University of
         Western Ontario}
\title{Power Simulations Comparing GVStat with LiMcLeod Portmanteau Tests}

\Plainauthor{A. Ian McLeod, Second Author} %% comma-separated
\Plaintitle{Power Simulations} %% without formatting

\Abstract{
The results reported by \citet[Figure 1 and Tables 1-2]{MahdiMcLeod2010} could be reproduced using the following simulation
function, \code{simpower}.
This function calls the function \code{onesim} which implements the parametric bootstrap test for one simulation.
One can run these two functions on a PC with single CPU using the argument \code{RmpiQ = FALSE} or
on a cluster computer with multiple CPU's using the argument \code{RmpiQ = TRUE} where the \pkg{Rmpi} package should be installed.
Instructions to install and run \code{Rmpi} package under Windows is given in the link
\url{http://www.stats.uwo.ca/faculty/yu/Rmpi/windows.htm}.
}

\Keywords{Parametric Bootstrap Significance Test, Portmanteau Test, \VARMA Models
}

\Address{
  A. Ian McLeod\\
  Department of Statistical and Actuarial Sciences\\
  University of Western Ontario\\
  E-mail: \email{aim@stats.uwo.ca}\\
  URL: \url{http://www.stats.uwo.ca/mcleod}\\
  Esam Mahdi\\
  Department of Statistical and Actuarial Sciences\\
  University of Western Ontario\\
  E-mail: \email{emahdi@uwo.ca}\\
}

%% end of declarations 
%%%%%%%%%%%%%%%%%%%%%%%%%%%%%%%%%%%%%%%%%%%%%%%
\def\VARMA{\textsc{varma}\,}
\def\VAR{\textsc{var}\,}
\def\VMA{\textsc{vma}\,}
\def\ARMA{\textsc{arma}\,}
\def\AR{\textsc{ar}\,}
\def\MA{\textsc{ma}\,}
\def\AIC{\textsc{aic}\,}
\def\MMC{\textsc{mmc}\,}  

\usepackage{Sweave}
\begin{document}


%\VignetteIndexEntry{Notes on rwm}
%\VignetteKeywords{PL Commands, decision support systems, ephemeral computing, workspace, teaching}
%\VignettePackage{rwm}



%%%%%%%%%%%%%%%%%%%%%%%%%%%%%%%%%%%%%%%%%%%%%%%%%%%%%%%%%%%%%%%%%%%%
\section[Power Simulations]{Power Simulations}
\label{PowerSimulations}

After loading the packages \pkg{portes} and \pkg{Rmpi} in \proglang{R} session,
users can copy the following two \proglang{R} scripts, \code{simpower} and \code{onesim}, 
then run them in \proglang{R} session.

\begin{Schunk}
\begin{Sinput}
R> library("portes")
R> library("Rmpi")
 \end{Sinput}
 \end{Schunk}

\begin{verbatim}
R> "simpower" <-
+  function(phi, theta, sigma, intercept = NA, n, lags = seq(5, 30, 5), 
+   p = 1, NREP = 1000, NumSim = 10000, statistic = c("GVStat", "LiMcLeod"),
+   StableParameters = NA, RmpiQ = FALSE, SquaredQ = FALSE, Trunc.Series = NA,
+   alpha = c(0.01, 0.05, 0.1), Set.Seed=TRUE){
+      statistic <- match.arg(statistic)
+      if (RmpiQ==FALSE){ 
+           if (Set.Seed) 
+               set.seed(321123789) 
+        sim.stat <- replicate(NumSim, onesim(phi = phi, theta = theta,
+         sigma = sigma, intercept = intercept, n = n, lags = lags,
+         p = p, NREP = NREP, statistic = statistic, StableParameters =
+         StableParameters, SquaredQ = SquaredQ, Trunc.Series = Trunc.Series))
+       }
+       else {       
+          mpi.spawn.Rslaves()
+          if (Set.Seed)
+             mpi.setup.rngstream(21597341)
+          mpi.bcast.Robj2slave(NumSim)
+          mpi.bcast.Robj2slave(NREP)
+          mpi.bcast.Robj2slave(phi)
+          mpi.bcast.Robj2slave(theta)
+          mpi.bcast.Robj2slave(sigma)
+          mpi.bcast.Robj2slave(intercept)
+          mpi.bcast.Robj2slave(n)
+          mpi.bcast.Robj2slave(lags)
+          mpi.bcast.Robj2slave(Trunc.Series)
+          mpi.bcast.Robj2slave(statistic)
+          mpi.bcast.Robj2slave(SquaredQ)
+          mpi.bcast.Robj2slave(blockToeplitz)
+          mpi.bcast.Robj2slave(GVStat)
+          mpi.bcast.Robj2slave(LiMcLeod)
+          mpi.bcast.Robj2slave(ImpulseVMA)
+          mpi.bcast.Robj2slave(InvertQ)
+          mpi.bcast.Robj2slave(simvarma)
+          mpi.bcast.Robj2slave(simvma)
+          mpi.bcast.Robj2slave(onesim)
+          if (all(!is.na(StableParameters))){
+                mpi.bcast.Robj2slave(StableParameters)
+                mpi.bcast.cmd(library(akima))
+                mpi.bcast.Robj2slave(interpp)
+                mpi.bcast.Robj2slave(interpp.old)
+                mpi.bcast.Robj2slave(MCTable3)
+                mpi.bcast.Robj2slave(MCTable4)
+                mpi.bcast.Robj2slave(MCTable5)
+                mpi.bcast.Robj2slave(MCTable7)
+                mpi.bcast.Robj2slave(FitStable)
+                mpi.bcast.Robj2slave(rstable)
+          }
+     if(statistic=="GVStat")
+      sim.stat <- mpi.parReplicate(NumSim, onesim(phi = phi, theta = theta, 
+       sigma = sigma, intercept = intercept, n = n, lags = lags, p = p,
+       NREP = NREP, statistic = "GVStat", StableParameters = StableParameters,
+       SquaredQ = SquaredQ, Trunc.Series = Trunc.Series))
+     else
+      sim.stat <- mpi.parReplicate(NumSim, onesim(phi = phi, theta = theta, 
+       sigma = sigma, intercept = intercept, n = n, lags = lags, p = p, 
+       NREP = NREP, statistic = "LiMcLeod", StableParameters = StableParameters, 
+       SquaredQ = SquaredQ, Trunc.Series = Trunc.Series))
+      mpi.close.Rslaves()
+     }       
+    m <- length(lags)
+    out <- matrix(numeric(m*length(alpha)),ncol=m,nrow=length(alpha))
+    for (i in 1: length(alpha))
+      out[i,] <- rowMeans(sim.stat<=alpha[i])
+   return(out)
+ }
\end{verbatim}

\begin{verbatim}
R> "onesim" <-
+   function(phi, theta, sigma, intercept = NA, n, lags = seq(5, 30, 5),
+       p = 1, NREP = 1000, statistic = c("GVStat", "LiMcLeod"), 
+       StableParameters = NA, SquaredQ = FALSE, Trunc.Series = NA){
+         statistic <- match.arg(statistic)
+          if (is.na(Trunc.Series)) 
+            Trunc.Series <- min(100,ceiling(n/3))
+          sigma <- as.matrix(sigma)
+          k <- NCOL(sigma)
+       sim.data <- simvarma(phi = phi, theta = theta, sigma = sigma, 
+        intercept = intercept, n = n, StableParameters = StableParameters,
+        Trunc.Series = Trunc.Series) 
+  if (all(is.na(intercept)))
+   fitvar1 <- ar.ols(sim.data, aic = FALSE, intercept = FALSE, order.max = p)
+  else
+   fitvar1 <- ar.ols(sim.data, aic = FALSE, intercept = TRUE, order.max = p)
+   res <- ts(as.matrix(fitvar1$resid)[-(1:p),])
+     if(statistic=="GVStat")
+       obs.stat<-GVStat(res, lags,p,SquaredQ)[,2]
+     else
+       obs.stat<-LiMcLeod(res, lags,p,SquaredQ)[,2]
+     count<-rep(0, length(lags))
+     for (i in 1:NREP){
+         sigma <- as.matrix(fitvar1$var.pred)
+          if (is.array(fitvar1$ar)){
+           arrayphi <- array(numeric(k * k * p), dim = c(k^2, p))
+            for (i in 1:p) arrayphi[, i] <- c(fitvar1$ar[i, , ])
+             phi <- array(c(arrayphi), dim = c(k, k, p))
+          }
+          else
+             phi <- fitvar1$ar
+         theta <- NULL
+         if (!is.null(fitvar1$x.intercept)) 
+             intercept <- fitvar1$x.intercept
+         else
+             intercept <- rep(0,k)
+        bootdata <- simvarma(phi = phi, theta = theta, sigma = sigma, 
+          intercept = intercept, n = n, StableParameters = StableParameters, 
+          Trunc.Series = Trunc.Series)
+        FitSimModel <- ar.ols(bootdata, aic = FALSE, intercept = FALSE, 
+                 order.max = p)
+        rboot <- ts(as.matrix(FitSimModel$resid)[-(1:p), ])
+     if(statistic=="GVStat")
+      sim.stat<-GVStat(rboot, lags,p,SquaredQ)[,2]
+     else
+      sim.stat<-LiMcLeod(rboot, lags,p,SquaredQ)[,2]
+     count <- count+(sim.stat>=obs.stat)
+   }
+   ans<-(count+1)/(NREP+1)
+   names(ans)<-lags
+   return(ans)
+ }
\end{verbatim}

where
\begin{description}
  \item[phi] is a numeric or an array of \code{AR} or an array of \code{VAR} parameters with order $p$.
  \item[theta] is a numeric or an array of \code{MA} or an array of \code{VMA} parameters with order $q$.
  \item[sigma] is the variance of white noise series and must be entered as matrix in case of bivariate or multivariate time series.
  \item[intercept] is the mean vector of the series.
  \item[n] is the length of the series.
  \item[lags] is the vector of lag values.
  \item[p] is the order of the fitted \code{VAR} model in simulation procedures. 
  \item[NREP] is the number of bootstrap replications.
  \item[NumSim] is the number of simulations.
  \item[statistic] is the generalized test, \code{"GVStat"}, or the modified test, \code{"LiMcLeod"}.  
  \item[StableParameters] is the four stable parameters, \code{ALPHA, BETA, GAMMA,} and \code{DELTA}.  
  If it is not \code{NA}, then the \proglang{R} package \pkg{akima} should be loaded.

\begin{Schunk}
\begin{Sinput}
R> library("akima")
 \end{Sinput}
 \end{Schunk}
   
  \item[SquaredQ] when it is \code{TRUE} then apply test to squared residuals values in simulation procedures.
  \item[Trunc.Series] is the truncation lag used to truncate the infinite \code{MA} or \code{VMA} Process.            
\end{description}


%%%%%%%%%%%%%%%%%%%
\subsection[Simulation Example Using Rmpi]{Simulation Example Using Rmpi}
\label{ExampleRmpi}

The performance of the generalized variance portmanteau test, \code{GVStat}, and its competitor,
\code{LiMcLeod}, were compared by the parametric bootstrap simulations.
We consider Model 1 from \cite{MahdiMcLeod2010} with length series, $n=100$,
$10^3$ simulations, and $10^2$ replications to get some timings.
The main simulations were run on a computer with double quad core CPU's using the \pkg{Rmpi} package \citep{YuRmpiRnews}.
The time was 25 minutes for the bootstrap generalized variance test as described in \cite{MahdiMcLeod2010} and 14 minutes for 
the bootstrap version of that one given by \citet{LiMcLeod1981}.

\begin{Schunk}
\begin{Sinput}
R> NREP <- 10^2
R> NumSim <- 10^3
R> k <- 2
R> sigma <- matrix(c(1, 0.71, 0.71, 1), k, k)
R> phi <- array(c(0.5, 0.4, 0.1, 0.5, 0, 0.3, 0, 0), dim = c(k,
+     k, 2))
R> theta <- NULL
R> lags <- seq(5, 30, 5)
R> Trunc.Series <- 50
R> intercept <- NA
R> n <- 100
R> alpha <- 0.05
R> Start1 <- proc.time()[3]
R> simpower(phi = phi, theta = theta, sigma = sigma, intercept = intercept,
+     n = n, lags = lags, p = 1, NREP = NREP, NumSim = NumSim,
+     statistic = "GVStat", Trunc.Series = Trunc.Series, alpha = alpha,
+     RmpiQ = TRUE, StableParameters = NA, SquaredQ = FALSE, Set.Seed = TRUE)
\end{Sinput}
\begin{Soutput}
        8 slaves are spawned successfully. 0 failed.
master (rank 0, comm 1) of size 9 is running on: stats-c-emim 
slave1 (rank 1, comm 1) of size 9 is running on: stats-c-emim 
slave2 (rank 2, comm 1) of size 9 is running on: stats-c-emim 
slave3 (rank 3, comm 1) of size 9 is running on: stats-c-emim 
slave4 (rank 4, comm 1) of size 9 is running on: stats-c-emim 
slave5 (rank 5, comm 1) of size 9 is running on: stats-c-emim 
slave6 (rank 6, comm 1) of size 9 is running on: stats-c-emim 
slave7 (rank 7, comm 1) of size 9 is running on: stats-c-emim 
slave8 (rank 8, comm 1) of size 9 is running on: stats-c-emim 
Loading required package: rlecuyer
      [,1]  [,2]  [,3]  [,4]  [,5]  [,6]
[1,] 0.686 0.557 0.463 0.394 0.351 0.329
R> End1 <- proc.time()[3]
R> Total1 <- End1 - Start1
R> Total1  
elapsed 
1502.99 
\end{Soutput}
\begin{Sinput}
R> Start2 <- proc.time()[3]
R> simpower(phi = phi, theta = theta, sigma = sigma, intercept = intercept,
+     n = n, lags = lags, p = 1, NREP = NREP, NumSim = NumSim,
+     statistic = "LiMcLeod", Trunc.Series = Trunc.Series, alpha = alpha,
+     RmpiQ = TRUE, StableParameters = NA, SquaredQ = FALSE, Set.Seed = TRUE)
\end{Sinput}
\begin{Soutput}
        8 slaves are spawned successfully. 0 failed.
master (rank 0, comm 1) of size 9 is running on: stats-c-emim 
slave1 (rank 1, comm 1) of size 9 is running on: stats-c-emim 
slave2 (rank 2, comm 1) of size 9 is running on: stats-c-emim 
slave3 (rank 3, comm 1) of size 9 is running on: stats-c-emim 
slave4 (rank 4, comm 1) of size 9 is running on: stats-c-emim 
slave5 (rank 5, comm 1) of size 9 is running on: stats-c-emim 
slave6 (rank 6, comm 1) of size 9 is running on: stats-c-emim 
slave7 (rank 7, comm 1) of size 9 is running on: stats-c-emim 
slave8 (rank 8, comm 1) of size 9 is running on: stats-c-emim 
      [,1]  [,2]  [,3] [,4]  [,5]  [,6]
[1,] 0.519 0.347 0.287 0.25 0.222 0.197       
R> End2 <- proc.time()[3]
R> Total2 <- End2 - Start2
R> Total2
elapsed 
 839.54
 \end{Soutput}
 \end{Schunk}
 
\bibliography{portesV}

\end{document}


